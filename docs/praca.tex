\documentclass{pracamgr}

\usepackage[polish]{babel}
\RequirePackage{longtable}
\usepackage[T1]{fontenc}
\usepackage[utf8]{inputenc}
\usepackage[pdftex]{color,graphicx}


\author{Michał Ziemba}

\nralbumu{237954}

\title{Serwis Submit++}

\tytulang{The Submit++ service}

\kierunek{Informatyka}

\opiekun{dr Aleksy Schubert\\
  Instytut Informatyki\\
  }

\date{Czerwiec 2012}

\dziedzina{ 
11.3 Informatyka
}

\klasyfikacja{D. Software}

\keywords{Serwis, zadanie, rozwiązanie, MIMUW}

\begin{document}
\maketitle

\begin{abstract}
  W~pracy przedstawiono opis i dokumentację narzędzia JBSC, służącego do
  analizy statycznej bajtkodu Javy. Jest to wtyczka do środowiska programistycznego
  Eclipse, korzystająca z frameworku Umbra. Umożliwia proste wykrywanie potencjalnie
  niebezpiecznych konstrukcji w kodzie bez jego uruchamiania.
\end{abstract}

\tableofcontents

\chapter*{Wprowadzenie}
\addcontentsline{toc}{chapter}{Wprowadzenie}

    Analiza statyczna kodu polega na badaniu własności programu bez jego uruchamiania,
    w przecwieństwie do analizy dynamicznej. 

    Motywacja
    Ogólnie pojęta analiza statyczna ma coraz większe zastosowanie w procesie wytwarzania wolnego od błędów
    i bezpiecznego kodu. Analiza statyczna kodu wykonywalnego ma zastosowanie w systemach osadzonych, gdzie
    nie ma dostępu do źródeł. Także w przypadku systemów których źródła nie są po prostu wolnodostępne.

Dłuższy opis działania
Java Bytecode Static Checker jest wtyczką do Eclipsa 

Opis rozdziałów

\chapter{Kod bajtowy}\label{r:bytecode}

    Kod bajtowy jest formą instrukcji, wykonywanych przez maszynę wirtualną Javy. Każdy kod instrukcji
    ma długość jednego bajta, przy czym niektóre instrukcje wymagają parametrów.

    Specjalne i zarezerwowane instrukcje:
    \begin{itemize}
        \item 
        \item
    \end{itemize}
    
    Instrukcje kodu bajtowego można podzielić na następujące grupy:
    \begin{itemize}
        \item
    \end{itemize}
    
    Przedrostki i przyrostki
    
    Generowanie kodu bajtowego, javac (przykład użycia javac -d}
    
    Przykład kodu bajtowego?

\chapter{Analiza statyczna}\label{r:staticanalysis}

    Analiza statyczna polega na badaniu kodu źródłowego bez jego uruchamiania. Pozwala wykryć potencjalnie
    niebezpieczne konstrukcje w kodzie. Jednak fakt niewykrycia błędu na poziomie analizy statycznej nie czyni
    kodu bezbłędnym.
    
    Możliwości i ograniczenia (zalety)
    Zalety korzystania z narzędzi do analizy statycznej:
    \begin{itemize}
        \item Sprawdzenia wykonywane przez narzędzie, w odróżnieniu od sprawdzenia programisty,
              pozwalają wykryć potencjalne błędy w miejscach mniej interesujących, które mogły zostać pominięte
              lub przeoczone przez programistę.
        \item Analiza statyczna daje możliwość wykrycia problemów u ich źródła. Pozwala uniknąć sytuacji,
              kiedy w trakcie uruchamiania programu otrzymujemy błąd przepełnienia buforu, a nie wiemy dokładnie
              kiedy, gdzie i dlaczego to przepełnienie nastąpiło.
        \item Błędy są znajdowane na wczesnym etapie rozwijania kodu. To daje możliwość uniknięcia być może
              kosztownego usuwania tego błędu na późniejszym etapie.
    \end{itemize}
    
    Najpoważniejszym zarzutem, który pojawia się pod adresem narzędzi do analizy statycznej jest fakt, że
    produkują zbyt wiele ostrzeżeń, z których większość jest nieprzydatnych. Takie fałszywe ostrzeżenia nazywamy
    \textit{false negatives}.
    
    Zakres analizy statycznej
    Przy pomocy analizy statycznej można wykonywać następujące sprawdzenia:
    \begin{itemize}
        \item 
    \end{itemize}
    
    Etapy analizy
    
    Co można sprawdzać
    
    False positives, false negatives
    
    Przykładowe narzędzia i ch krótki opis:
    \begin{itemize}
        \item lint
        \item findbugs
        \item escjava2
    \end{itemize}
    
    
\chapter{Integracja}\label{r:integration}

    Co to jest Umbra?

    Połączenie z frameworkiem Umbra

\chapter{Funkcjonalność systemu}\label{r:funkcjonalnosc}

\section{Wprowadzenie}


\chapter{Środowisko systemu}\label{r:srodowisko}

\section{Serwer}


\chapter{Budowa systemu}\label{r:budowa}

\section{Architektura systemu}




\chapter{Podsumowanie}\label{r:podsumowanie}


\appendix

\chapter{Dostarczone dokumenty}

Obiekty dostarczone w ramach pracy licencjackiej to:
\begin{itemize}
\item dokument z treścią pracy licencjackiej
\item płyta CD
\end{itemize}

\chapter{Spis zawartości płyty CD}




\begin{thebibliography}{99}
\addcontentsline{toc}{chapter}{Bibliografia}


\end{thebibliography}


\end{document}

